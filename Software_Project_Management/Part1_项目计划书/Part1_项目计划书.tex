\documentclass[UTF-8]{ctexart}
\usepackage{graphicx} % Required for inserting images
\usepackage{geometry}
\geometry{left=2.5cm,right=2.5cm,top=2.5cm,bottom=2.5cm}

\title{软件项目管理}
\author{2127405048方浩楠}
\date{\today}

\begin{document}

\maketitle

\tableofcontents
\newpage

\section{项目建议书}
现在高校普及因此上大学的学生越来越多,因此高校宿舍管理的要求也越来越高,原本的宿舍管理方式面对如今越来越大的信息量难以招架,加上现在的孩子对于学校原有的管理方式个性化要求越来越高,因此为了满足现今的需求,需要更加先进的方式进行管理。因此依据现有的信息技术的基础上开发出校园宿舍管理系统。这个系统可以将所有学生基本信息、宿舍情况、宿舍分配及特殊情况等全部记录在系统中,方便宿舍管理员查询管理。校园宿舍管理系统的应用,将原本的宿舍管理数据庞大及个性化需求等各方面问题都相应的解决了。

学生宿舍管理系统是一个旨在提供高效、便捷和安全宿舍管理的软件系统。下面详细描述学生宿舍管理系统的需求和目标:

\subsection{学生宿舍管理系统的需求}

\subsubsection{学生信息管理}

系统应能够存储和管理所有学生的基本信息,包括姓名、学号、联系信息等。此外,还需包括学生的入住记录和历史信息。

\subsubsection{宿舍的合理分配和管理}

系统应自动分配宿舍给学生,并支持手动分配。宿舍分配应基于一系列因素,如性别、年级、特殊需求等。管理员应能够轻松地查看和修改宿舍分配。

\subsubsection{宿舍的维修请求}

学生和宿舍管理员应能够提交和跟踪宿舍维修请求。系统应分配任务给维修人员,并记录维修历史。

\subsubsection{费用和账单管理}

系统应记录学生的住宿费用,生成账单,并支持在线支付。宿舍管理员应能够监控费用未付款的情况。

\subsubsection{报表和统计分析}

系统应能够生成各种报表和统计数据,如入住率、维修请求处理时间、费用收入等,以帮助宿舍管理决策和监督。

\subsection{学生宿舍管理系统的目标}
本系统旨在实现下述的功能

\begin{enumerate}
    \item 提高宿舍管理效率
    \item 提供便携的学生体验
    \item 提高学生宿舍管理系统的安全性
    \item 实现财务透明,为学生提供方便的在线充值渠道
    \item 数据分析和决策支持,提供详细的数据报表和统计分析,以帮助宿舍管理人员做出更明智的决策,优化资源分配和预测需求。

\end{enumerate}

总之,学生宿舍管理系统的需求和目标旨在改善宿舍管理流程,提供更好的服务,提高效率,并确保安全和合规性。这将有助于提高学校宿舍管理的质量,提供更好的学生体验。

\subsection{学生宿舍管理系统的主要功能}
作为一个学生宿舍管理系统,本系统需要拥有以下几种功能:
\begin{enumerate}
    \item 学生管理:添加学生、删除学生、修改学生信息、查询学生,使用Excel批量导入学生,指定学生宿舍。
    \item 分配宿舍:批量为指定的学生指定宿舍。
    \item 调整宿舍:学生可以发起调整宿舍申请,管理员同意后,即可更换宿舍。
    \item 批量办理退寝:管理员可以批量为学生办理退寝。
    \item 退寝审批:学生如果需要退寝,可以填写退寝表,管理员同意后,即可退寝。
    \item 设备报修单:学生如果需要对损坏设备进行报修,可以在本系统进行设备报修登记,然后再由维修员进行处理。
    \item 离校登记:节假日或其他原因,学生需要暂时离校,即可直接在本系统进行离校登记,管理员可以进行查询。
    \item 宿舍管理:添加宿舍、删除宿舍、修改宿舍信息、查询,筛选宿舍。
    \item 楼栋管理:添加楼栋、删除楼栋、修改楼栋信息、查询,筛选楼栋。
    \item 账号管理:本系统所有的账号都是由管理员进行添加的,管理员可以在本系统对账号进行添加、查询、修改、删除。

\end{enumerate}

\subsection{学生宿舍管理系统的特点}
作为一个新型的系统,本系统需要有下述特点:
\begin{enumerate}
    \item 用户友好性: 界面应简单易用,以确保学生和管理员可以轻松浏览和使用系统。
    \item 实时更新: 提供实时信息更新,确保学生和管理员能够获得最新的宿舍相关信息。
    \item 安全性: 该系统需要采取必要的安全措施,以保护学生和宿舍信息的隐私和安全,以免遭受网络攻击
    \item 可扩展性: 具备扩展功能的能力,以便将来根据需要添加新功能。
    \item 可定制性: 允许学校根据自身需求进行定制,包括添加自定义字段或流程,并且可以自定义功能
    \item 数据备份和恢复: 定期备份数据,以防止数据丢失,并提供数据恢复功能,保证在遭受灾害时可以恢复数据以免数据丢失
    \item 性能优化: 学生宿舍人数众多,因此需要确保系统在高负载时仍然能够快速响应。
\end{enumerate}

\subsection{学生宿舍管理系统的规模和期限}

\subsubsection{项目规模}

本项目的开发规模大约为10人月,即需要10个人工作一个月的时间来完成。本项目的代码量大约为1万行,其中包括前端、后端、数据库等部分。本项目的文档量大约为50页,其中包括需求规格说明书、设计文档、测试文档、维护文档等。

\subsubsection{项目期限}

本项目的开发周期为3个月,本项目采用敏捷开发方法,按照迭代和增量的方式,将项目分为6个迭代。每个迭代的开始和结束都有一个里程碑,用于确定和交付相应的功能模块和文档。每个迭代的具体内容如下:
\begin{itemize}

\item 迭代一:完成需求分析和系统架构设计,并编写需求规格说明书和系统架构设计文档。
\item 迭代二:完成数据库设计和宿舍分配功能的开发和测试,并编写数据库设计文档和宿舍分配功能的代码和测试文档。
\item 迭代三:完成界面设计和宿舍调换功能的开发和测试,并编写界面设计文档和宿舍调换功能的代码和测试文档。
\item 迭代四:完成宿舍查询功能和宿舍考核功能的开发和测试,并编写宿舍查询功能和宿舍考核功能的代码和测试文档。
\item 迭代五:完成宿舍报修功能和宿舍统计功能的开发和测试,并编写宿舍报修功能和宿舍统计功能的代码和测试文档。
\item 迭代六:完成系统集成测试、性能测试、安全测试,并编写系统集成测试报告、性能测试报告、安全测试报告。对用户进行系统培训,并收集用户反馈,进行系统优化和改进,并编写培训文档和优化文档。

\end{itemize}

\subsection{学生宿舍管理系统的市场前景和经济效益}

\subsubsection{市场前景}

软件项目管理是指对软件项目的各个阶段进行有效的计划、组织、协调、控制和评估,以保证软件项目按照既定的目标、范围、质量、成本和时间完成。软件项目管理是软件工程的重要组成部分,也是软件产业发展的关键因素。随着数字化转型、云计算、人工智能等新技术的不断发展和应用,软件项目管理的需求和挑战也日益增加,软件项目管理软件作为一种提高软件项目管理效率和质量的工具,具有广阔的市场前景。根据1,2023年中国项目管理软件市场规模预计将达到120亿元,年均复合增长率为\textbf{15\%}。根据2,2023年全球项目管理软件市场规模预计将达到67.5亿美元,年均复合增长率为\textbf{9.4\%}。本项目作为一个学生宿舍管理系统软件项目,旨在提高学生宿舍管理的效率和质量,满足学生的住宿需求和期望,提升学生的安全感和满意度。本项目的目标用户群体为高校学生、宿舍管理员、教务处等,预计有较大的市场潜力和竞争优势。

\subsubsection{经济效益}

本项目的经济效益主要体现在以下几个方面:

\begin{enumerate}

\item 降低学生宿舍管理成本:通过使用本项目开发的学生宿舍管理系统软件,可以实现对学生宿舍的信息化和智能化管理,减少人工干预和错误,提高管理效率和精准度,节省人力、物力、财力等资源。
\item 增加学生宿舍管理收入:通过使用本项目开发的学生宿舍管理系统软件,可以实现对学生宿舍的考核、报修、统计等功能,提高学生宿舍的卫生、安全、纪律等水平,提升学生宿舍的品牌形象和口碑,吸引更多的学生入住,增加学生宿舍的入住率和收益。
\item 创造附加价值:通过使用本项目开发的学生宿舍管理系统软件,可以实现对学生宿舍的查询、调换等功能,满足学生的个性化需求和偏好,提升学生的住宿体验和满意度,增强学生的归属感和忠诚度,促进学生之间的交流和互助,培养学生的团队精神和社会责任感。

\end{enumerate}

\end{document}
